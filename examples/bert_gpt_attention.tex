\begin{figure}[htbp]
    \centering
    \begin{tikzpicture}[
        node distance=0.8cm and 0.8cm,
        font=\footnotesize,
        >=Latex,
        token/.style={circle, draw=gray!60, fill=gray!5, thick, minimum size=0.8cm},
        target/.style={circle, draw=blue!60, fill=blue!5, thick, minimum size=0.8cm},
        arrow/.style={-Latex, gray!50},
        connect/.style={-Latex, thin, gray!40}
    ]
        % BERT (Left)
        %\node[font=\bfseries] at (2.5, 4.5) {BERT: 双向注意力 (Bidirectional)};

        \foreach \i in {1,...,4} {
            \node[token] (b_in_\i) at (\i*1.2, 0) {$x_\i$};
            \node[target] (b_out_\i) at (\i*1.2, 3) {$h_\i$};
        }

        % Connections for BERT (All-to-All)
        \foreach \i in {1,...,4} {
            \foreach \j in {1,...,4} {
                \draw[connect] (b_in_\i) -- (b_out_\j);
            }
        }

        % GPT (Right)
        \begin{scope}[xshift=7cm]
             %\node[font=\bfseries] at (2.5, 4.5) {GPT: 因果注意力 (Causal)};

            \foreach \i in {1,...,4} {
                \node[token] (g_in_\i) at (\i*1.2, 0) {$x_\i$};
                \node[target] (g_out_\i) at (\i*1.2, 3) {$h_\i$};
            }

            % Connections for GPT (Causal: j depends on i where i <= j)
            \foreach \i in {1,...,4} {
                \foreach \j in {\i,...,4} {
                    \draw[connect] (g_in_\i) -- (g_out_\j);
                }
            }
            \end{scope}

            \node[anchor=north] at (6, -0.5) {\footnotesize $x$:输入词元;$h$:输出隐状态。连线表示注意力(Attention)信息流动方向};

        \end{tikzpicture}
        \caption{BERT双向注意力与GPT因果注意力机制对比(左:BERT;右:GPT)}
        \label{fig:bert_gpt_arch}
\end{figure}
